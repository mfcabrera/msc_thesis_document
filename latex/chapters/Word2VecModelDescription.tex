\chapter{Description of the Word2Vec Model }
\label{chap:word2vec_description}


\section{Introduction}
As described in  previous chapter,  continuous language space models have
demonstrated good results across a variety of tasks. Word vector
representation are in some way a byproduct of \ac{NNLM}, as not only the
model itself is learned by the neural network but also the word representation.  As pointed out by
the authors, these continuous word vector representation capture syntactic
and semantic relationship of words not possible to achieve from a discrete
representation of traditional $n$-gram models. Furthermore, these
representation can be used to improve existing \ac{NLP} tasks.

One of the way to learn a \ac{NNLM} is to learn the word vector using a
\ac{NN} with a single hidden layer. Then, these word vectors are used to
train the \ac{NNLM}. In these architecture learns the word vectors without
even constructing the complete language model.  This chapter describes in
detail a recent approach  to efficiently learn
high-dimensional word representation \cite{DBLP:journals/corr/abs-1301-3781}
focusing only in the first part, that is, learning good word representation
representations.

Another interesting architecture of NNLM was presented in [13, 14], where the word vectors are
first learned using neural network with a single hidden layer. The word vectors are then used to train
the NNLM. Thus, the word vectors are learned even without constructing the full NNLM. In this
work, we directly extend this architecture, and focus just on the first step where the word vectors are
learned using a simple model.

\section{Description of The Model}


\section{Optimization Strategies used by Word2vec}
\label{sec:strategies_improve_time}

\subsection{Huffman Coding}
\label{sec:huffman_coding}

Huffman coding is an entropy  encoding algorithm used for lossless data
compression.  It was developed by David A. Huffman while he was a Ph.D.
student at MIT. \cite{huf52}.

Huffan cioding is an algorithm to proce an  optimal symbol code $C$ for a given 
\emph{pmf} $ p = (p_1,....,p_n) $.  Huffman coding builds a tree orderded by frequency.



%%% Local Variables: 
%%% mode: latex
%%% TeX-master: "../main.tex"
%%% End:
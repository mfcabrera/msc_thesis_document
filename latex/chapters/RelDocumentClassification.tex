
\section{Text Categorization}
\label{sec:rel_text_categorization}

The goal of \ac{TC} is to automatically assign, for each
documents, categories selected among a predefined set.
In opposition to the machine learning typical multi-class classification
task which aims at attributing one class among the possible ones to each
example, documents in a text categorization task may belong to several
categories. 
%Sebastiani Ind
% "Sebastiani02"


Formally,  \ac{TC} is the  task of  assigning a bolean value to each pair 
$\left\langle d_{j},c_{j}\right\rangle \in\mathcal{D}\,\, x\,\mathcal{\, C}$
where  $\mathcal{D}$  is a domain of documents and
$\mathcal{C}=\{c_{1},\ldots,c_{|c|}\}$ is a set of predefined categories. A
true value assigned to the pair $\left\langle d_{j},c_{j}\right\rangle $
indicates the decision of filling  $d_{j}$ under the category  $c_{j}$, and
a false value the decision of not filling it under that category \cite{Sebastiani02}. 

More formally the task is to approximate a unknown target function 
 $\breve{\Phi}:\mathcal{D}\,\, x\,\mathcal{\, C}\rightarrow\{T,F\}$  using 
 $\Phi:\mathcal{D}\,\, x\,\mathcal{\, C}\rightarrow\{T,F\}$  such that 
 $\Phi$ and  $\breve{\Phi}$ ``coincide as much as possible'' \cite{Sebastiani02}. 

So far only the binary case has been discussed, however in many realistic
settings there are many possible (mutually exclusive or not) categories in
which a document can be classified. In these stitutation 

 There are two main ways to  tackle
such problem: 1-vs-All and All-vs-All. 

\textbf{TODO: INSERT THE INFORMATION FROM SCIKIT-LEARN PAGES - e.g.
  DESCRIPTION OF 1-vs-All and All vs all}

\ac{TC} relies  heavily on  models that have been developed originally for \ac{IR}.
 the reason of this is that \ac{TC} is a content based document management task,
 and such it shares many characteristics with other IR tasks such as text
 search, specifically \ac{IR}-style document representation  using \ac{BOW} model described in section \ref{sec:rel_bow}.



\subsection{Performance Measures}
\label{sec:sub_performance_measures}

One question that arises after reading the definition of TC is how to measure
 the how well the classifier function  $\breve{\Phi}$ (the approximation or hypothesis) matches the unknown target function  $\Phi$
 (effectiveness).  This is generally measured in terms of the classic IR
 metrics of precision  $P$  and recall  $R$ but adapted to the specific case
 of \ac{TC}. $P$  is defined as the probability that if a given a random document
 $d_{x}$  is classified under the category  $c_{i}$  the classification is
 correct:

 $$P(\breve{\Phi}(d_{x},c_{i})=T\,\,|\,\, P(\Phi(d_{x},c_{i})=T))$$


 Analogously  $R$  is defined as the probability that if a random document 
 $d_{x}$ is ought to be classified under  $c_{i}$,  this decision is taken
 by the classifier function: 

 $$P(\Phi(d_{x},c_{i})=T\,\,|\,\, P(\breve{\Phi}(d_{x},c_{i})=T))$$
 



%%% Local Variables: 
%%% mode: latex
%%% TeX-master: "../main.tex"
%%% End:
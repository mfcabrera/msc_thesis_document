
\section{Text Categorization}
\label{sec:rel_text_categorization}

The goal of \ac{TC} is to automatically assign, for each
documents, categories selected among a predefined set.
In opposition to the machine learning typical multi-class classification
task which aims at attributing one class among the possible ones to each
example, documents in a text categorization task may belong to several
categories. 
%Sebastiani 
% "Sebastiani02"

Formally,  \ac{TC} is the  task of  assigning a bolean value to each pair 
$\left\langle d_{j},c_{j}\right\rangle \in\mathcal{D}\,\, x\,\mathcal{\, C}$
where  $\mathcal{D}$  is a domain of documents and
$\mathcal{C}=\{c_{1},\ldots,c_{|c|}\}$ is a set of predefined categories. A
true value assigned to the pair $\left\langle d_{j},c_{j}\right\rangle $
indicates the decision of filling  $d_{j}$ under the category  $c_{j}$, and
a false value the decision of not filling it under that category \cite{Sebastiani02}. 

More formally the task is to approximate a unknown target function 
 $\breve{\Phi}:\mathcal{D}\,\, x\,\mathcal{\, C}\rightarrow\{T,F\}$  using 
 $\Phi:\mathcal{D}\,\, x\,\mathcal{\, C}\rightarrow\{T,F\}$  such that 
 $\Phi$ and  $\breve{\Phi}$ ``coincide as much as possible'' \cite{Sebastiani02}. 

So far only the binary case has been discussed, however in many realistic
settings there are many possible (mutually exclusive or not) categories in
which a document can be classified.  There are two main ways to  tackle
such problem: 1-vs-All and All-vs-All. 

WRITE HERE ABOUT THE

%%% Local Variables: 
%%% mode: latex
%%% TeX-master: "../main.tex"
%%% End:
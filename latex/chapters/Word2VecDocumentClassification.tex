\chapter{Word2vec for Document Classification of German language business document}
\label{chap:rel_word2vec_doc_classification}

% Introduction 
%  - Promises of deep learning
%  - Help many task
%  - Document classification is an interesting problem because there is a
%  standard methdology to use word vector representation
%  Structure of the chapter
%  Gini document infraestructure (qucik description of Count Dooku)
%  Gini dataset and document classification dataset.
%  Description of the current implementation
%  BoW word model
%  Word2Vec document mean vector representation
%  Comparisson of the model
%  Different Feature visualization
%  Conclusion and Future work.
%   - Using individual features will make always hard to work with variable
%   - Sized documents - that is definitive.

\section{Introduction}
\label{sec:w2v4tc_intro}

One of the motivation of feature learning is to be able to
obtain useful representation that will improve existing task. In the case of
\ac{NLP} word vector features have been shown to improve existing taks such
as \ac{NER}, Chunking and Sentiment Analysis \cite{Turian:2010:WRS:1858681.1858721}
\cite{DBLP:journals/corr/abs-1103-0398}  and have shown promising results in
fields as machine translation and speech processing \cite{collobert:2008}
\cite{DBLP:journals/corr/MikolovLS13}.  

In the field of \ac{TC} there have
not been much work of improving this task by using word vector representations. There are many reason
fo this. One can be that many of the current methods work well enough, in particular those
based on the \ac{VSM} \cite{Sebastiani02}. Another reason might be that that
when using word representation each word get a $n$-dimensional vector
representation representation, therefore a large document could be only
represented as big matrix,  and current methods work only with vector
representation od document. However there has been previous work using word
vector for short text classification, namely sentiment analysis
\cite{maas2011learning} . Sentiment Analysis can be seen as a variation of
\ac{TC}. In fact in the aforementioned work the use the traditional
approaches for \ac{TC} to train the sentiment predictor.  In this work the
performance  word vector representations based on the \textit{Word2vec} model 
are compared against other features in the context of classifying  business-related German language
documents. 

The rest of the chapter is divided as follows: The first part describes 
the company's  document platform, the requirements and challenges behind this
particular document classification task. The second the part describes the different features
used and the classification algorithms used while explaining the limitation
faced.  The third and final  part discusses the most important results as
well as well as the future work. 

\section{Gini document platform}
\label{sec:gini_doc_platform}

Gini currently own a database of mostly German language document. These
are personal and business related documents that have been bought 
from  persons all over Germany. Document type are really broad and go from
long insurance contract to short remittance slips. These document are stored
in a central system database that can be accessed via an  \textsc{API}.
Document can be either \textit{native} or \textit{scanned}. The document
\textit{native} that are native are orignal digital document. (e.g. PDF document) that
are uploaded to the system. \textit{Scanned} documents on the other hand, are
uploaded from an image obtained a scanner or a camera (e.g. a mobile phone
camera). Given the variability of the capture devices for the scanned
documents, the quality of such images is not regular among all the documents.
This is important to consider, given that after being uploaded all the
documents are processed using a \ac{OCR} solution, so the quality of the
images affect the the accuracy with which the text is extracted from. 
After the text has been extracted, it is stored  along with
the position of the words a into a  database using a custom
\textsc{XML} format. 

% - Stored Dookud
% - Stored as DocXML not only text but also location information of words.
% - Two ways to get the text from the document {Native, Scanned (e.g.)}


%  approach as well as the most
% important results. It includes  a performance comparison of the word vectors in German with their counterpart in English as well as  the evaluation of  some corpus
% preprocessing approaches that might affect the performance of the word
% representation for German.





%%% Local Variables: 
%%% mode: latex
%%% TeX-master: "../main.tex"
%%% End: